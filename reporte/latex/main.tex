\documentclass[10pt,conference,letterpaper]{IEEEtran}
\usepackage{util/estilo}

% Title and authors
\title{
    \rule{\linewidth}{2pt}
    \Title{Estrategias de optimización en compiladores de lenguajes funcionales modernos} \\[-1ex]
    \rule{\linewidth}{2pt}
}

\author{
    \IEEEauthorblockN{Santiago Romero}
    \IEEEauthorblockA{Estudiante de Ciencias de la Computación\\
    Facultad de Ciencias UNAM, Ciudad de México\\
    \email{santiago.rp@ciencias.unam.mx}}
    \and
    \IEEEauthorblockN{Arlet Pinacho}
    \IEEEauthorblockA{Estudiante de Ciencias de la Computación\\
    Facultad de Ciencias UNAM, Ciudad de México\\
    \email{arlet.pinacho@ciencias.unam.mx}}
    \and
    \IEEEauthorblockN{Juvenal Guzmán Condado}
    \IEEEauthorblockA{Estudiante de Ciencias de la Computación\\
    Facultad de Ciencias UNAM, Ciudad de México\\
    \email{juvenal.guzman@ciencias.unam.mx}}
    \and
    \IEEEauthorblockN{Ricardo Iván Martínez Cano}
    \IEEEauthorblockA{Estudiante de Ciencias de la Computación\\
    Facultad de Ciencias UNAM, Ciudad de México\\
    \email{ricardoivan@ciencias.unam.mx}}
}

% Document starts
\begin{document}

\maketitle

\thispagestyle{plain}
\pagestyle{plain}

\begin{abstract}
    Los compiladores de lenguajes funcionales modernos enfrentan el doble
    desafío de preservar las abstracciones de alto nivel mientras ofrecen una
    ejecución eficiente en arquitecturas de bajo nivel. En este reporte
    investigamos técnicas de optimización para compiladores funcionales
    mediante el diseño e implementación de \emph{subs}, un lenguaje funcional
    que demuestra una compilación en múltiples etapas. Subs se traduce al
    cálculo lambda, el cual posteriormente se convierte en un \textit{cálculo
    de combinadores SKI extendido} enriquecido con combinadores adicionales que
    permiten una optimización y ejecución práctica y eficiente. A partir de
    esta representación intermedia, los programas se compilan hacia dos
    entornos distintos en C++: uno \emph{estricto} y uno \emph{perezoso}. El
    runtime perezoso aprovecha la técnica de \emph{graph reduction} para
    modelar la evaluación no estricta, mientras que el runtime estricto
    proporciona características de rendimiento predecibles. Nuestro trabajo
    ilustra cómo los supercombinadores pueden servir como una representación
    intermedia práctica y optimizable, conectando los fundamentos teóricos y
    prácticos del campo de los compiladores de lenguajes funcionales.
\end{abstract}

\begin{IEEEkeywords}
    SKI, Combinadores, Supercombinadores, Cálculo lambda, Optimización de código,
    Lenguajes funcionales, Graph reduction, Lambda lifting, Generación de código,
    Compiladores, Lenguajes de programación
\end{IEEEkeywords}

\section{Preliminares}\label{sec:preliminares}
\subfile{subs/preliminares.tex}

\section{Introducción}\label{sec:intro}
\subfile{subs/introduccion.tex}

\section{Metodología}\label{sec:metodologia} %chance esto no debería llamarse metodología pero no sé que poner
\subsection{Cálculo Lambda}\label{ssec:lambda}
\subfile{subs/metodologia/lambda.tex}

\subsection{Combinadores SKI}\label{ssec:ski} %chance esto podría llamarse de una mejor forma
\subsubsection{¿Que es SKI?}\label{sssec:ski-que-es}
\subfile{subs/combinadoresSKI.tex}
\subsubsection{SKI como lenguaje de compilación intermedio}\label{sssec:ski-intermedio}
\subfile{subs/SKIintermedio.tex}

\subsection{Generación de código}\label{ssec:generacion}
\subsubsection{Cálculo Lambda a Cálculo SKI}\label{sssec:lambda-ski}
\subfile{subs/lambda-SKI.tex}

\subsubsection{Cálculo SKI a C++}\label{sssec:ski-c}
\subfile{subs/SKI-C.tex}

\subsection{Optimizaciones}\label{ssec:optimizaciones}
\subsubsection{Optimizaciones propias de SKI}\label{sssec:opt-ski}
\subfile{subs/metodologia/optimizaciones/opt-ski.tex}

\subsubsection{Graph reduction}\label{sssec:graph-red}
\subfile{subs/metodologia/optimizaciones/graph-red.tex}

\subsubsection{Lambda Lifting}\label{sssec:lambda-lift}
\subfile{subs/lambda-lifting.tex}

\subsection{Implementación}\label{ssec:implementacion} % ¿esto es necesario?
\subfile{subs/implementacion/overview.tex}

\section{Resultados y Discusión}\label{sec:resultados} % ¿esto es necesario?
\subfile{subs/resultados.tex}

\section{Conclusión}\label{sec:conclusion}
\subfile{subs/conclusion.tex}

%\section*{Agradecimientos}
%This research was supported by Example Funding Agency. The authors would also
%like to thank the Example Lab for their assistance with data collection.

\printbibliography

\end{document}
