\documentclass[main.tex]{subfiles}
\usepackage{util/estilo}

\begin{document}
Este módulo define una gráfica dirigida como una lista de vértices y una
función de adyacencia que dado un vértice $a$, regresa un conjunto
(HashSet) de vértices adyacentes.

Además, establece las siguientes funciones para manipular gráficas dirigidas:
\begin{itemize}
\item \boldit{stronglyConnected}: Dada una gráfica y una lista
  de vértices devuelve el conjunto de los vértices que son
  alcanzables a partir del conjunto inicial. Para ello realiza un recorrido
  DFS.
  
\item \boldit{converse}: A partir de una gráfica dirigida permite
  construir la gráfica dirigida con el mismo conjunto de vértices
  pero con un conjunto de flechas en el que se conservan las relaciones
  de la original pero con sentido contrario.

\item \boldit{acyclicSort}: Asumiendo que la gráfica de entrada es acíclica,
  se realiza un recorrido DFS y cada vértice se inserta en la lista de
  salida \textbf{sólo después de haber procesado recursivamente todos sus vecinos}.
  Esto produce un ordenamiento topológico de los vértices, sin duplicados.
  
\item \boldit{acyclicOrder}: Se basa en \textit{acyclicSort} para ordenar
  los vértices de la gráfica y les asigna un índice que marca
  su posición en el orden topológico.
  
\item \boldit{subdfs}: Es una función auxiliar que hace DFS a partir de un
  vértice $v$ y devuelve una tupla cuyo primer elemento es el conjunto de
  vértices alcanzados con $v$ y el conjunto de los vértices vistos inicialmente
  más los que se vieron durante el DFS realizado.
  
\item \boldit{stronglyConnectedComponents}: Regresa una lista con las componentes conexas
  de la gráfica dirigida. Para ello, ordena los vértices con \textit{acyclicSort} y por
  cada uno si no ha sido visto hace DFS en la gráfica con las flechas invertidas,
  recuperando en cada aplicación el conjunto de vértices descubierto.
  Es una implementación del algoritmo de Kosaraju
  \parencite[p.~617]{cormen2009introduction} para componentes fuertemente
  conexas.
  
\item \boldit{verticesInCycles}: Dada una gráfica dirigida permite encontrar los
  vértices que están dentro de un ciclo dirigido o bien, que tienen un lazo.
  
  Para ello, se obtienen las componentes conexas de la gráfica y se revisa: Si
  solo constan de un elemento se revisa si tienen un lazo, si constan de más
  vértices entonces dichos vértices están en un ciclo pues de no ser así no
  sería posible alcanzar a alguno de los vértices desde otro, por lo que no
  sería una componente conexa.
\end{itemize}
\end{document}
