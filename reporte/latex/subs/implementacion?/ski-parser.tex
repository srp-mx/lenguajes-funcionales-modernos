\documentclass[main.tex]{subfiles}
\usepackage{util/estilo}

\begin{document}
Este componente define el funcionamiento de un analizador sintáctico para el
lenguaje $SKI$. Para se especifican las siguientes funciones:

\begin{itemize}
\item \boldit{parserInfo}: Permite obtener la información que identifica al
  analizador sintáctico, concretamente su nombre.

\item \boldit{parser}: Ejecuta el analizador léxico del lenguaje y
  analiza su salida para reportar un error léxico o bien, para iniciar el
  análisis sintáctico mediante la construcción del árbol de sintáxis
  abstracta.

\item \boldit{parseTok}: Dada una lista de tokens, la expresión
  construido hasta el momento y la profundidad de parentizado
  esta función recorre la lista de tokens y uno a uno hace lo siguiente.
  
  Si apenas se va a construir la primera expresión ya sea del programa o
  dentro de un paréntesis:
  \begin{itemize}
  \item Si es alguno de los combinadores ($S$, $K$ o $I$), construye el
    combinador mediante el constructor de \textit{Expr} respectivo y
    aplica recursión con el resto de la lista de tokens.
    
  \item Si es una cadena aplica \textit{parseStr} pues es necesario
    extraer la cadena sin las secuencias de escape y luego contruirla
    mediante el constructor \textit{Str} del tipo \textit{Expr}. Después,
    aplica recursión con el resto de la lista de tokens.

  \item Si es un paréntesis que abre, se revisa que haya algo después,
    y se aplica recursión con el resto de tokens y con un
    nivel más de profundidad.

  \item Si es un paréntesis que cierra, es un error.
  \end{itemize}

  Si ya se construyó alguna expresión y lo que sigue en la lista:
  \begin{itemize}
  \item Es alguno de los combinadores ($S$, $K$ o $I$) o una cadena,
    se construye una aplicación  en la que la expresión construida
    previamente es la función y el nuevo combinador, o bien,
    la cadena extraída mediante la función \textit{parseStr}
    es el argumento. Esta aplicación es usada para la
    recursión con el resto de la lista de tokens.

  \item Si es un paréntesis que abre, se revisa que haya algo después,
    y se aplica recursión con el resto de tokens y con un
    nivel más de profundidad.

  \item Si es un paréntesis que cierra, es un error.
  \end{itemize}
\end{itemize}

\end{document}
