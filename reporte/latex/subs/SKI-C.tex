\documentclass[main.tex]{subfiles}
\usepackage{util/estilo}

\begin{document}
Con el fin de poder obtener los archivos ejecutables de nuestros programas,
decidimos implementar un módulo encargado de generar el programa equivalente
en C++ para entonces aprovechar el compilador G++. Tenemos dos implementaciones
de este proceso, una que lleva a cabo una
evaluación estricta o ansiosa y otra que permite la evaluación perezosa pero
ambas siguen una idea similar.

La idea consiste en los siguientes puntos:
\begin{itemize}
\item Definir una estructura $Value$ en C++ que puede contener una cadena o una
  función, 
\item Función $Apply$ que dadas dos instancias $f$ y $x$ de tipo $Value$
  permite aplicar el primero al segundo según sea el caso:
  \begin{itemize}
  \item Cadena y función $\rightarrow$ evalúa la función con la cadena vacía
    como parámetro y si el resultado es una cadena, la concatena a la primera.
    \item Cadena y cadena $\rightarrow$ concatena ambas cadenas.
    \item Función y cualquier valor $\rightarrow$ aplica la función al valor.
    \item Otro caso $\rightarrow$ ocurre un error.
  \end{itemize}
\item Implementar el comportamiento de los combinadores $S$, $K$, $I$, $B$, $C$
  y algunas variantes $Bs$, $Cp$ y $Sp$.

\item Gracias a estas definiciones lo que sigue es reemplazar las
  expresiones del programa fuente a su definición en C++ que construimos
  con anterioridad, valor que almacenamos en una variable
  \textit{ProgramExpr}.

\item Se define el método \textit{main} en el que se aplica el programa
  a cada argumento de línea de comandos mediante la función \textit{Apply}
  y se imprime el resultado.  
\end{itemize}

La diferencia con la versión de evaluación perezosa es que se incorpora una
estructura \textit{Thunk} que no es más que una función cuyo tipo de retorno
es \textit{Value} y que tiene un método \textit{force} el cual
debe ser llamado para que la función se evalúe. Esta estructura se usa
para definir que una función del lenguaje recibe un \textit{Thunk} lo que
permite a las funciones decidir cuándo evaluar su argumento.
\end{document}
