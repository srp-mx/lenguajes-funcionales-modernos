\documentclass[main.tex]{subfiles}
\usepackage{util/estilo}

\begin{document}

En esta sección se presentan las definiciones y convenciones que se emplearán a
lo largo del trabajo. El objetivo es establecer un marco común que permita la
formulación rigurosa de los resultados posteriores.
\begin{itemize}
    \item \textbf{Términos Lambda}: Decimos que los Términos Lambda son palabras que se definen en el siguiente alfabeto\parencite{The-Lambda-Calculus-its-Syntax-and-Semantics}:
    \begin{itemize}
        \item Variables: $v_0,v_1,\cdots$
        \item Abstractor: $\Lambda$
        \item Parentesis: $(.)$
    \end{itemize}
    Utilizamos las siguientes reglas para definir al conjunto de $\lambda$-Terminos de manera inductiva\parencite{The-Lambda-Calculus-its-Syntax-and-Semantics}:
    \begin{itemize}
        \item $x \in \Lambda$: Una variable $x$ es una cadena que representa a un parámetro, por si misma puede ser un término lambda.
        \item $M \in \Lambda \Rightarrow (\lambda x M) \in \Lambda$: Una abstracción lambda del tipo $(\lambda x.M)$ es una función la cuál toma como entrada a la variable $x$ y regresa al cuerpo $M$ como resultado.\\
        Decimos que esta definición de función es un término lambda.
        \item $M,N \in \Lambda \Rightarrow (MN) \in \Lambda$: Una aplicación $(MN)$ es la aplicación de una función $M$ a un argumento $N$ dónde $M$ y $N$ son términos lamda.
    \end{itemize}
    \item \textbf{Gráfica}: Una gráfica es un conjunto de puntos conocidos como vértices, mientras que las lineas que conectan a estos puntos se conocen como aristas.\parencite{Introduction-To-Graph-Theory-Wilson}\\
    Decimos que las aristas pueden representar relaciones entre los vértices que estas conectan.\parencite{Introduction-To-Graph-Theory-Wilson}
    \item \textbf{DAG}: Una gráfica dirigida acíclica (DAG por sus siglas en inglés), es una gráfica en dónde cada una de sus aristas cuenta con una dirección en la que pueden ser recorridas.\parencite{Acyclic-Digraphs-Bang-Jensen-Gutin}
    Sumado a esto decimos que esta gráfica es acíclica debido a que no cuenta con ciclos.\parencite{Acyclic-Digraphs-Bang-Jensen-Gutin}
    \item \textbf{Índices de Brujin}: Son una técnica que nos permite escribir expresiones del cálculo lambda en dónde a cada variable se le reemplaza por un número que representa la cantidad
    de niveles en la expresión que debemos de tomar ``saltar'' para encontrar a la variable con la que se encuentra ligada.\parencite{The-Lambda-Calculus-its-Syntax-and-Semantics}
  \item \textbf{Efectos secundarios}: Un lenguaje de programación permite
    la presencia de efectos secundarios si una o más variables cambian su valor
    durante la ejecución de una función dentro de la cuál no fueron definidas, es
    decir, no son variables locales \cite{chakrabarti1985states}.
\end{itemize}
\end{document}

