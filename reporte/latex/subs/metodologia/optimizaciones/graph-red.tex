\documentclass[main.tex]{subfiles}
\usepackage{util/estilo}

\begin{document}
    Graph Reduction es una técnica de evaluación utilizada comunmente en lenguajes funcionales, tales como Haskell. Se puede 
    considerar el equivalente gráfico a la reducción dentro del cálculo lambda y permite el uso de funciones de alto orden y 
    evaluación perezosa.

    En las implementaciones de graph reduction las expresiones a evaluar se van a representar dentro de su forma de árboles de 
    sintaxis, donde las hojas van a representar los valores constantes, como lo son integers, chars y basadoooleans; las funciones 
    predefinidas, como lo son las operaciones aritmeticas o booleanas; y los nombres de las variables. Se utiliza el símbolo '@' 
    como una indicación de que el nodo es la aplicación de una función. 
    
    \Tree[.@ [f [.x ]]]
    \begin{center}
         \textit{Árbol de sintaxis para f(x)}
    \end{center}

    \Tree[.@ [.@ [.a ][.+ ]] [.b ]]
    \begin{center}
         \textit{Árbol de sintaxis para a+b}
    \end{center}
    
    

\end{document}
