\documentclass[main.tex]{subfiles}
\usepackage{util/estilo}

\begin{document}

%% TODO: Esto es del proyecto anterior

\PARstart{L}os lenguajes de programación constituyen especificaciones
mediante las cuales los programadores pueden definir secuencias de tareas
a ejecutar por una computadora, llamadas programas, comúnmente
haciendo uso de un lenguaje más cercano al entendimiento humano. 
Por ello, se requiere que los programas pasen por un proceso de traducción
que permita obtener una representación entendible y ejecutable por la máquina.

Llamamos \textbf{compilador} al software encargado de realizar
esta traducción, compuesta por distintas fases que permiten comprender
la estructura del \textbf{programa fuente} y transformarla sin dejar de
asegurar que la nueva representación, el \textbf{programa objeto},
sea correcta, eficiente y que conserve su significado original. 

Generalmente, este proceso se divide en dos grandes fases: la fase de
\textit{análisis} y la fase de \textit{síntesis}. La fase de análisis
se encarga de procesar el programa fuente para asegurar su correctitud
y definir de manera unívoca su estructura lógica mediante una
\textbf{representación intermedia}; por otro lado, la fase de síntesis
utiliza dicha representación para generar el programa objeto equivalente.

La fase de análisis está compuesta por tres etapas: 
\textit{análisis léxico}, \textit{análisis sintáctico} y 
\textit{análisis semántico}, cada una diseñada para abordar un nivel 
específico de la comprensión del programa fuente.

En este proyecto nos centraremos en la etapa del 
\textbf{análisis léxico}, cuyo objetivo es reconocer los \textbf{componentes
léxicos} o \textit{tokens} que constituyen el programa fuente. 
Para ello, implementaremos en \textit{Haskell} un analizador léxico
construido desde cero, aplicando los principios teóricos y algoritmos
de los \textbf{autómatas finitos} y \textbf{lenguajes formales}.
Dicha implementación busca no sólo validar la teoría, sino también
fortalecer la comprensión práctica de los procesos de traducción
en compiladores.

\end{document}

